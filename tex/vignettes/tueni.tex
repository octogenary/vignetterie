Le décès de Gebran Tuéni.

Gebran Tuéni, Ghassan Tuéni, Liban
===
\begin{quote}
    Il y a des hommes que l’épreuve grandit, au point qu’ils en deviennent immenses et qui, avec des mots simples et dignes, touchent les cœurs pour toujours. En prenant le micro devant la foule en pleurs, Ghassan Tuéni est entré définitivement dans la légende. « Il est rare, a-t-il dit, la voix juste un peu enrouée, qu’un homme fasse ses adieux à son fils à l’endroit où, soixante ans auparavant, il avait fait ses adieux à son père. Je me souviens comment Gebran Tuéni est mort foudroyé le 11 janvier 1947, alors qu’il prononçait un discours dans lequel il défendait l’unité de la Palestine et son arabité, ainsi que l’adhésion du Liban à la cause arabe. Il m’avait laissé ce message et, à mon tour, j’ai repris le flambeau et inculqué ces principes à mon fils. Celui-ci a suivi la même voie et scandé ces mêmes idées dans ce serment devenu un slogan repris par toute une génération de jeunes. » « Je dis cela, car je pense souvent à ce que vous dites, Monseigneur (Khodr, archevêque grec orthodoxe du Mont-Liban) : “La mort est une résurrection et Jésus est ressuscité des morts.” Aujourd’hui, vous avez affirmé que Gebran est parti pour nous préparer une place à la noce finale. Dois-je aller à sa rencontre ?… » « Aujourd’hui, je voudrais que les haines ainsi que les mots qui divisent soient enterrés en même temps que Gebran. Je n’appelle ni à la vengeance, ni à la rancœur, ni au sang. Je voudrais au contraire que nous reprenions d’une seule voix ce serment qu’il avait lancé à la place des Martyrs le jour de l’intifada 2005, dont il est la victime : J’appelle tous les Libanais, chrétiens et musulmans, à rester unis au service du Liban, cette grande patrie, et au service de sa cause arabe. »
\end{quote}
\nocite{2005b}

Apropos, Charles \textcite{glass2005} that same year.
\begin{quote}
    When the historian Kamal Salibi was 17, he watched the French army’s reluctant retreat from Lebanon. Under the Sykes-Picot Agreement with Britain in 1916, France had assumed a mandate, later ratified by the League of Nations, to govern Syria and Mount Lebanon. Its mission civilisatrice to the Christians of Lebanon led it to expand the borders of the Christian statelet, incorporating so many Muslims – both Sunni and Shia – from outside the Ottoman governorate of Mount Lebanon that the Muslims inevitably became a majority. The Sunnis in Damascus and the Druze in southern Syria revolted against French rule again and again, and the French bombarded Damascus and the Druze villages. By the time the French departed, even the Christians were in the streets demanding that they leave. ‘The French left very nicely on the last day of 1946,’ Salibi recalled, sitting in his West Beirut flat near the American University where he taught for forty years. ‘The Lebanese gave them a 21-gun salute. They were thanked for what they did for the country. The ugly side of the mandate was quickly forgotten.’ Only a compromise – the unwritten National Pact that distributed government offices by religious sect – saved Lebanon from fratricidal violence in 1946. Every sect took its share of the spoils: from the presidency, the prerogative of the Maronite Catholics, through the Sunni prime ministership and the Shia office of house speaker, down to the lowliest post in the civil service. Now, almost sixty years later, Salibi, the author of the standard history of Lebanon – A House of Many Mansions – was watching the Syrian army pack up and go home across the French-created border. There was no 21-gun salute. ‘If they had withdrawn gracefully of their own accord, they would have left with some courtesy and perhaps some gratitude. Instead, they left like housebreakers.’ To pre-empt embarrassing televised scenes of toppling statuary, the Syrian troops took with them imposing statues of two men: Hafez al-Assad, the Syrian president who sent his army into Lebanon in 1976; and his heir, Bashar al-Assad, who under international pressure brought the troops back in time for the United Nations deadline of 30 April.
\end{quote}