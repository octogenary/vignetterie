My thesis, explained.

vulgarisation, philo, logic, absolute generality, quantification, Graham Priest, dialetheism, Timothy Williamson, James Studd, set theory, expressibility, en cours
===
I propose to explain a little of my thesis \parencite{loo2024} in as close to plain language as possible.

If I say that ‘everything is in the suitcase’, it is natural to think that ‘everything’ is restricted \parencite[this example is drawn from][]{williamson2003}. In this case, the \emph{domain} of `everything' is \emph{restricted}, in a way determined by the \emph{context}: we might paraphrase the sentence as ‘everything mutually understood as to be packed has been packed’, or similar. We call this phenomenon \emph{quantifier domain restriction}. Linguists and philosophers of language are sometimes interested in how contextual domain restriction works: when does it happen? what are the relevant restrictions? and so on \parencite{stanley2000}. That is not what this thesis is about.

In some cases, we might not think that the word ‘everything’ is intended to be read as restricted in the same way. For example: `everything is subject to the power of God’; `science can discover the causal laws governing everything’; and ‘everything is physical’.