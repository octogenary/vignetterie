Circular and miscellaneous walks thence and possibly thither.

walks, Oxford
===

I have found the following to be feasible; in determining the applicability of that datum, it may be noted that I am not particularly athletic, although walking is not terribly strenuous since I am not particularly big.

About four or five miles: from central Oxford to the towpath, turn away from Port Meadow towards University Parks, thence down Mesopotamia and through the verge of Cowley back to the High Street.

A reasonable approximation of a half marathon proceeds from the High Street down St Aldate’s to the river, turns right and follows the river before following Castle Mill Stream and the canal path until Godstow; turn left to the Isis, and follow it until the path finally meets Wytham woods, before following the Green Belt Way in a loop back to the Thames path; follow that southwards opposite Port Meadow, and through Osney, before breaking off at St Aldate’s again, and returning to the city centre (perhaps the High Street). The path is not very well-maintained beyond Port Meadow.

About nine miles: down the towpath to Port Meadow, following the foot of which one reaches the Thames Path—continue to Wolvercote, turn right through the urban fringe and along the A40 briefly before a rather obscure turn leads to the Cherwell, leading eventually to University Parks and a choice of Mesopotamia or a more direct route.

About twenty-five miles: as above to the Thames Path, which should be followed until Farmoor Reservoir, whence Cumnor, I think Boars Hill and Kennington, and then back down the Thames to central Oxford.

About forty miles: down the Thames Path to Reading, with a stop at a pub for lunch; take the train back from the station.