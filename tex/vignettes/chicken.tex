A chicken stew.

recettes
===
Some years ago, \textcite{ottolenghi2017} published three recipes for slow-cooked chicken in The Guardian. My mother and I have revised the second.

The ingredients we use are nearly entirely the same, except that we usually forget the bay leaves, often use more chicken, and sometimes use different pasta. At some point we started to add celery, which I think is a good choice. There is therefore much more vegetable in the broth than in Ottolenghi’s variant. I am often lazy and use frozen sofrito; about 750–1000 grammes of it for 1.5 kg of chicken works for dinner party catering.
\begin{enumerate}
\item    Warm a large casserole on the stove or in the oven.
\item    Soften first the garlic, anchovies, shallots, and onions (unless the latter are to come from frozen sofrito) in the stove, and then celery and carrot (or frozen sofrito if using). I have obtained the best results by softening them in the oven at about 110C for half an hour, browning the base slightly; the vegetables should be introduced at the earliest after five minutes or so of softening, and preferably later.
\item    Pour in the vegetable stock, the herbs, and then the chicken. Leave for a suitable period (a note on timing follows).
\end{enumerate}

\section*{Timing.}
\begin{enumerate}
    \item I have with success left this recipe in the oven at ~120C (fan) for about eight hours, taking the lid off maybe an hour before serving to slightly speed the reduction of the sauce. The chicken will brown after long enough, and is pleasantly tender.
    \item There are some subtleties here: chicken not near the surface will not brown. This can be fixed by removing the chicken, which is particularly convenient e.g. if cooking for a small gathering with the intention of retaining substantial leftovers.
    \item Given less time, one can leave the lid off. One of course then has to check when the sauce is too evaporated, avoid burning, and so on.
    \item The shortest I have left this recipe in the oven is about two hours, at 160C.
\end{enumerate}

\section*{Serving.}
Flatbread or pasta (the original suggestion is tagliatelle) go well.

\section*{Leftovers.}
This keeps as long as any other chicken dish will keep in the fridge. It can also be frozen and microwaved relatively profitably.