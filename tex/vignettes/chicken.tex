A chicken stew.

recettes
===
Some years ago, \textcite{ottolenghi2017} published three recipes for slow-cooked chicken in The Guardian. My mother and I have revised the second.

The ingredients we use are nearly entirely the same, except that we usually forget the bay leaves, often use more chicken, and sometimes use different pasta. At some point we started to add celery, which I think is a good choice. There is therefore much more vegetable in the broth than in Ottolenghi’s variant. I am often lazy and use frozen sofrito; about 750–1000 grammes of it for 1.5 kg of chicken works for dinner party catering.
\begin{enumerate}
\item    Warm a large dutch or casserole on the stove or in the oven.
\item    Soften first the garlic, anchovies, shallots, and onions (unless the latter are to come from frozen sofrito) in the stove, and then celery and carrot (or frozen sofrito if using). I have obtained the best results by softening them in the oven at about 110C for half an hour, browning the base slightly; the vegetables should be introduced at the earliest after five minutes or so of softening, and preferably later. The herbs should also be thrown in at this point.
\item    Put the vegetable stock in, and then the chicken. Leave in the oven for two or three hours.
\item    Some of the chicken will have browned because it was exposed. Remove all the chicken, and return the chicken that hasn’t been browned into the oven on a separate tray.
\item    At the same time, reduce the sauce (without the chicken) either in the oven or the stove.
\item    When serving with pasta, ladle some of the sauce out of the casserole and into the pasta to mix, and mix with the rocket.
\end{enumerate}

The principal difference from Ottolenghi’s recipe is that the chicken is browned in the oven rather than seared.