‘Conundrum of visual perception’

Kyrgyzstan, flags, Sadyr Japarov
===
\begin{quote}


    “There are lot of people who hold the view that the current flag looks like a sunflower. And because of that, our state has been unable to rise up since it could do nothing but gaze upon the sun,” [the President, Sadyr] Japarov told a journalist[.]

    Where the rays in the older design were soft and wavy, they are now straight and spiky.

    As the word began spreading that somebody might have composed a flag with the wrong number of rays — and the detail is important as the number of rays is meant to represent the 40 Kyrgyz tribes — officials scrambled to limit the damage.

    Sure enough, on the night of January 3, Bishkek municipal workers were filmed lowering the flag and counting the rays up to the requisite number.

    That has not convinced the doubters, who suspect the wrong flag was simply substituted for a corrected version under the cover of night to enable this piece of choreography.

    It appears that subsequent to lawmakers giving their blessing to a new flag, the government tinkered further, thereby defying parliament’s prerogative to make final decisions on such matters.

    This all means at least four — not to say almost certainly more — types of Kyrgyz flags in total exist out in the world: the original, parliament’s version, the government’s version, and, it appears, a defective 39-rayed variant of the government’s version.

    “The new flag should have been raised in the presence of a guard of honor. At least in the presence of the secretary of state. And the national anthem should have been played. (Only an enemy force sneaks in and swaps the flag),” he [Tazabek Ikramov, a lawmaker] wrote on Facebook.

    Japarov’s team has grown weary of all this. The head of the presidential administration, Kanybek Tumanbayev, has called for punishments for anybody spreading what he termed “fake information” about the number of rays on the flag.

    Tumanbayev is concerned, among other things, that continued discussions on this topic are going to embarrass Kyrgyzstan.

    “Not only does such false information create discontent among many people, but now our Kazakh neighbors are discussing it too,“ he moaned.

    Human rights activist Gulshayir Abdirasulova wrote on January 4 that the State Committee for National Security, or \textsc{gknb}, the successor agency to the \textsc{kgb}, has been summoning people grumbling about the 39 rays matter for questioning.
\end{quote}

\nocite{imanaliyeva2024}