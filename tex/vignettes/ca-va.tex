À nous qui demandons aux Libanais « comment ca va ? »

Liban, L’Orient–Le Jour
===
\begin{quote}
    À chaque fois, c’est la même consternation, la même désolation, le même « je suis désolé(e) de ce qui vous arrive ». La même manière de nous regarder, avec parfois les yeux baissés et les mains jointes, avec la même pitié et le même embarras de ceux qui ont eu la chance d’être nés du bon côté de la planète. La chance de n’avoir jamais rencontré le désastre, le vrai. La chance de n’avoir jamais goûté à l’odeur de la poussière et du sang. La chance de n’avoir jamais été réveillés en pleine nuit par le chuintement d’un avion ou l’explosion d’un missile qui déchire sa ville.

    …Quelques minutes plus tard, avec une ponctualité démoniaque, viennent les routes bondées de gens qui fuient avec des morceaux de leurs vies d’avant dans des valises de fortune, puis les explosions, les nuages de fumée, les enfants arrachés à leur sommeil par le bruit des avions, les parents à bout qui ne savent plus comment les consoler et ceux qui sont loin qui les appellent frénétiquement pour s’assurer qu’ils sont encore en vie. Voilà comment ça va, au Liban, de ce mauvais côté du monde, où nos vies sont des poussières sur la balance des vies humaines…
\end{quote}
\nocite{khoury2024}