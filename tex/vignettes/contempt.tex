‘Ahem! Your Excellency, you are in contempt of court.’

Uganda, rule of law
===
So writes \textcite{ssemekadde2024} on Twitter.

\textcite[258]{widner2008} describe the Ugandan position thus.
\begin{quote}
     The way in which the judges conducted themselves differed too. One might not be entirely happy with the intellectual elegance, or lack thereof, of some of the Uganda opinions. No doubt there are times when a judge might be criticized for having given a party too much latitude. But there is sometimes wisdom in issuing messy judgments that create a principle or set a standard, on the one hand, and set aside immediate and drastic action, on the other. Some of the Uganda judgments had that character. By letting cases go ahead despite ambiguous charges while issuing a passport to the defendant, drawing out proceedings until election tempers cooled, allowing a result to stand while condemning violations of the law, etc., the Ugandan courts may have been able to build strength gradually, develop public understasnding and norms, and induce some parts of the government and opposition to trust their actions, even if the results were far from perfect from the standpoint of justice. Although certain government actions provoked condemnation from senior judges, many of the statements were coupled with phrasing designed to calm tempers. The Zimbabwe judgments, which were often elegant, tended to be more hard-hitting, and sharp public statements by judges in the years before the clash may have aggravated relationships. Human rights lawyers might disagree with this assessment, but it is always well to remember that the U.S. Supreme Court issued the murky Marbury in a charged atmosphere and exercised restraint subsequently, gradually building a reservoir of trust.
\end{quote}
