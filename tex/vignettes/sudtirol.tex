Multilingualism in Trentino–Alto Adige/Südtirol.

plurilinguisme, Italy, Italian, German
===
Italy has two levels of regional government: \emph{regioni}, and \emph{province}. Trentino–Alto Adige/Südtirol is a bilingual \emph{regione}, comprising the \emph{province} of Trentino and Alto Adige/Südtirol.

On entering the \emph{regione} on the A22 (Autobrennero/Brennerautobahn), some bilingualism is introduced. Place names are all in Italian, but other words also appear in German. The electronic signs will warn of \emph{Staugefahr} ending at \emph{Bolzano}. But in Alto Adige/Südtirol, the signs are all fully bilingual. On the \emph{Autostrada}, Italian seems to come first, but off it, German does. Thus \emph{strade provinciali} (\emph{Landesstraße}) are labelled ‘LS/SP’ (in that order) on signs indicating the road number. North of the border with Trentino, someone has to remember to change the place names as well as the other words: thus the \emph{Staugefahr} ends at \emph{Bozen}, not Bolzano.

Some roads are closed to certain vehicles, pursuant to
\begin{quote}
    Landesgesetz Nr.10 - 8.5.1990

    Legge prov.le n.10 - 8/5/1990[.]
\end{quote}

Most road names are translated too (thus \emph{Kirchweg}/\emph{Via della Chiesa}). Sometimes people can’t be bothered: thus, \href{https://maps.app.goo.gl/TRsLnrE4dhKzfsH96}{‘In der Lahn’}. Similarly, \emph{Fischerweg}/\emph{via dei Pescatori}. On the other hand, there are such absurdities as \href{https://maps.app.goo.gl/my7E59X28LyVMTeY9}{\emph{Grasweg}/\emph{vicolo Gras}}.