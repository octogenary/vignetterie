A letter on binding.

London Review of Books, bookbinding, libraries, printing, Hyphen Press, Hume, Oxford
===
The original read thus.
\begin{quote}
    Robin Kinross modestly omits to mention that he himself has had ‘books printed and bound elsewhere…by firms that still understand the workings of paper and glue’, so I hope you will forgive the unprompted advertisement for Hyphen Press \parencite{2024i}. Alas, neither British printers nor large publishers have been shamed into following his example. So what can be done?

    Nearly two decades ago, Frank \textcite{kermode2007} [in these pages] complained about poor binding, and Kinross pointed out the problem with hot-melt \parencite{2007a}. But even if reviewers were to complain in each case, readers would have to withhold their custom to create more than a marginal market for properly bound books.

    Consortia of academic libraries, on the other hand, have power to jointly withhold custom and large budgets; they have forced concessions on open access and publishing fees. But they seem content to buy even reference works that are barely usable when new and will be difficult to repair after heavy use.

    Durably bound books comprise multiple ‘sections’ of sheets folded in half, sewn together and cold glued. Unless the thread fails, a page would have to be torn completely to fall out. When a book is damaged, the thread can be removed, individual pages repaired, and sections resewn. (Hot-melt is difficult to remove from sections without tearing the paper.) Many old books in university libraries will have been rebound thus several times.

    I have on my desk a copy of \emph{Oxford Handbook of Hume} from the Bodleian. It is a ‘perfect’ binding: the pages are simply stacked on top of each other and glued together at the back, and fall out easily when the glue fails; it is difficult to reattach them properly. The glue of course is hot-melt; it is only because the book is so thick that it will lie flat (and only in the middle pages). It is £157.50 new.
\end{quote}

I should add that readers also command some institutional organisation, and that we too therefore share part of the blame. I am not aware of a single discussion in a library committee in which the quality of bindings was raised. Indeed, even the prospect of raising such a matter or proposing to take any action of the sort I propose above would seem fogeyish. (I should be happy to be corrected.) For this, we get the glue we deserve.

\nocite{2024h}