The <em>Circulaire Peeters</em>.

Belgique, plurilinguisme, français, néerlandais
===
\begin{quote}
    Sinds 1963 wordt het faciliteitensysteem op eenzelfde wijze toegepast: wanneer een Franstalige particulier eenmaal aan de administratie een Franstalig document had aangevraagd, bleef hij/zij dit document automatisch in het Frans ontvangen. …De regering wees erop dat de faciliteiten beperkend moeten worden geïnterpreteerd als tijdelijke overgangsmaatregelen, teneinde het de Franstalige inwoners in de Rand mogelijk te maken zich te integreren in het Vlaamse gebied. En daarom moet een Franstalige versie van de documenten telkens opnieuw worden aangevraagd.

    (Machine translated:) Since 1963, the facilities system has been applied in the same way: once a French-speaking individual had requested a French-language document from the administration, he/she automatically continued to receive this document in French. …The government pointed out that the facilities should be interpreted restrictively as temporary transitional measures, in order to enable French-speaking residents in the Rand to integrate into the Flemish area. And that is why a French-language version of the documents must be requested each time.
\end{quote}
\nocite{depre2001}