Décret 71.

tchad, plurilinguisme, arabe, français
===
\textcite{coudray}---
\begin{quote}    
L’un de ces moments de crise aura été, incontestablement, après le débat sur le statut de la langue arabe à la \textsc{cns}, celui où fut institué le bilinguisme dans le système scolaire tchadien. C’est le fameux « Décret 71 », stipulant que, désormais, « l’arabe et le français sont les langues d’enseignement dans les établissements publics ». Aussitôt, ce fut une levée générale de boucliers: l’Imam, au nom du Conseil Supérieur des Affaires Islamiques, dénonça dans cette décision « un Plan programmé pour anéantir les Ecoles Arabes ». À l’opposé, le parti Tchad-Avenir, constitué de cadres du Sud, affirma que le « bilinguisme n’est qu’un cheval de Troie, l’objectif final étant l’instauration d’une République islamique ». Comment avait-t-on pu en arriver là ? Un titre intelligent et malicieux du journal Le Progrès nous aidera à le comprendre : « Bilinguisme. Les madrassa vont (beaucoup) se franciser et les écoles vont (un peu) s’arabiser ».
\end{quote}